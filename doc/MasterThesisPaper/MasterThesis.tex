\documentclass[12pt, oneside]{report}
%\documentclass[12pt, twoside, openany]{report}
%% -------- packages and configuration --------

\usepackage{a4wide}
\usepackage{amsfonts}
\usepackage{amsmath}
\usepackage{enumerate}
\usepackage{verbatim}
\usepackage[T1]{fontenc}
\usepackage[utf8]{inputenc}
\usepackage[MeX]{polski}
\usepackage{amssymb, latexsym}
\usepackage{amsthm}
\usepackage{palatino}
\usepackage{array}
\usepackage{pstricks}
\usepackage{textcomp}
\theoremstyle{definition}
\newtheorem{theorem}{Twierdzenie}[section]
\newtheorem{remark}{Uwaga}[section]
\newtheorem{definition}{Definicja}[section]
\newtheorem{alg}{Algorytm}[chapter]
\newtheorem{prz}{Przypadek}[section]
\newtheorem{np}{Przykład}[section]
\newtheorem{lemma}[theorem]{Lemat}
\newcommand*{\norm}[1]{\left\Vert{#1}\right\Vert}
\newcommand*{\abs}[1]{\left\vert{#1}\right\vert}
\newcommand*{\om}{\omega}
\usepackage{geometry}
\geometry{left=25mm, right=25mm, bindingoffset=10mm, top=25mm, bottom=25mm}
\usepackage{graphicx}
\graphicspath{ {Images/} }
\usepackage[english, polish]{babel}
\usepackage{lmodern}
\usepackage{float}
\usepackage{verbatim}
\usepackage{setspace} 
%\onehalfspacing


\author{Jakub Abelski}
\title{Opracowanie symulatora transportera wahadła odwróconego na wózku}

\begin{document}

%% -------- title page --------
\begin{titlepage}
\pagestyle{empty}
\noindent
\begin{Large}
%\begin{table}[t]
%\centering
%\begin{tabular}[t]{lcr}
% \includegraphics[width=70pt,height=70pt]{PW} & POLITECHNIKA WARSZAWSKA & \includegraphics[width=70pt,height=70pt]{MiNI}\\
%& WYDZIAŁ MATEMATYKI & \\
%& I NAUK INFORMACYJNYCH &
%\end{tabular}
%\end{table}

\begin{center}
\begin{tabular}{lcr}
	\centering
	\begin{tabular}{c}
		\includegraphics[width=70pt,height=70pt]{PW}
	\end{tabular} &
	\begin{tabular}{c}
		\small 
		POLITECHNIKA WARSZAWSKA 
		\vspace*{5mm} \\
		\small
		WYDZIAŁ MATEMATYKI \\
		\small
		I NAUK INFORMACYJNYCH 
	\end{tabular} &
	\begin{tabular}{c}
		\includegraphics[width=70pt,height=70pt]{MiNI}
	\end{tabular}
\end{tabular}
\end{center}

\vfill
\begin{center}PRACA DYPLOMOWA MAGISTERSKA\end{center}
\begin{center}INFORMATYKA\end{center}
\end{Large}

\linespread{1.5}
\begin{center}
\Huge
\textbf{Opracowanie symulatora transportera wahadła odwróconego na wózku}
\end{center}

%\begin{center}
%\Large
%\textbf{Opracowanie symulatora transportera wahadła odwróconego na wózku}
%\end{center}

\vfill
\begin{center}
\Large
Autor:\\
\LARGE Jakub Abelski
\end{center}
\vfill
\begin{center}
\Large
Promotor: prof. dr hab. Krzysztof Marciniak
\end{center}
\vfill
\begin{center}
\large
Warszawa, Czerwiec 2016
\end{center}

%% -------- title page reverse --------
\newpage
\hfill
\begin{table}[b]
\centering
\begin{tabular}[t]{ccc}
............................................. & \hspace*{100pt} & .............................................\\
podpis promotora & \hspace*{100pt} & podpis autora
\end{tabular}
\end{table}
\end{titlepage}

%% -------- abstract --------
\setlength{\parindent}{5ex}
\selectlanguage{polish}

\begin{abstract}
\end{abstract}

\newpage
\pagestyle{empty}
\vspace*{\fill}
\begin{center}
\LARGE\textbf{Podziękowania}\\
\end{center}

\vspace{\fill}

%% -------- table of contents --------
\newpage
\pagestyle{plain}
\setcounter{page}{3}
\tableofcontents

%% -------- chapter I --------
\newpage
\pagestyle{headings}
\hyphenation{Syl-ves-tra}
\hyphenation{Syl-ves-ter-a}

\chapter{Wstęp}
\section{Motywacja i cele aplikacji}


%% -------- chapter II --------
\newpage
\chapter{Definicja projektu}
\section{Zakres projektu}
\section{Charakterystyka użytkownika}
\section{Analiza wymagań}
\subsection{Ogólne wytyczne}
\subsection{Funkcjonalności}
\subsection{Ograniczenia}


%% -------- chapter III --------
\newpage
\chapter{Opis technologii}
\subsection{Porównanie rozwiązań}

%% -------- chapter IV --------
\chapter{Architektura systemu}
\section{Wprowadzenie}
\section{Ogólny opis rozwiązania}
\section{Komponenty aplikacji}
\section{Opis wizualny aplikacji}


%% -------- chapter V --------
\newpage
\chapter{Dokumentacja techniczna}
\section{Informacje ogólne} 
\section{Opis techniczny modułów}


%% -------- chapter VI --------
\newpage
\chapter{Podsumowanie projektu}
\section{Instrukcja użytkownika}
\section{Opis testów}
\subsection{Testowane elementy programu}
\subsection{Wyniki testów}
\section{Ocena rozwiązania}
\subsection{Stopień realizacji projektu}
\subsection{Poprawność rozwiązania}

	
%% -------- chapter VII --------
\newpage
\chapter{Podsumowanie pracy}
\section{Wpływ projektu na autora}
\section{Krytyczna refleksja}
\section{Możliwości rozszerzania projektu}

	
%% -------- bibliography --------
\pagestyle{plain}
\begin{thebibliography}{11}

\bibitem{B} Stanisław Białas, \emph{Macierze. Wybrane problemy}, AGH Uczelniane Wydawnictwa Naukowo-Dydaktyczne, Kraków, 2006.

\bibitem{VuforiaFeatures} \emph{Vuforia Features},
\\*
https://www.qualcomm.com/products/vuforia/features

\end{thebibliography}


%% -------- statement --------
\selectlanguage{polish}
\clearpage
\pagestyle{empty}
\noindent Warszawa, dnia \today
\vspace{5cm}
\begin{center}
	\LARGE{Oświadczenie}
\end{center}
Oświadczam, że pracę magisterską pod tytułem: ,,Opracowanie symulatora transportera wahadła odwróconego na wózku'', której promotorem jest prof. dr hab. Krzysztof Marciniak, wykonałem samodzielnie, co poświadczam własnoręcznym podpisem.
\vspace{2cm}
\begin{flushright}
...........................................
\end{flushright}

\end{document}

